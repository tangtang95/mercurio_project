\section{Introduction}
Mercurio is a project whose objective is to design and develop an integrated and modular system that draws information from various sources and uses it to predict the happening of out-of-the-ordinary financial events. To this aim the system integrates both time-dependent and highly frequent numerical data (e.g. price, volume) and textual data (e.g. financial news articles, financial breaking news).
Data exposed by the sources are used to detect significant financial events that are either key-events (i.e. they convey considerable changes of the financial market) or signals (i.e. symptoms anticipating a key-event). Event recognition strategies vary depending on the type and nature of the managed data. The recognized events determine a temporal sequence of happenings that is then used to construct models to predict the happening of key-events in particular. To this aim sequential pattern mining techniques are applied to construct a model composed of temporal patterns. The constructed model and real-time data are used to provide users with alerts such as “there is a certain probability that company A will encounter key event C within X timeslots”. \\
Until now, the Mercurio project has analyzed and considered data regarding the italian market and the companies that operates in it. The contribution that have been brought with our work regards, in the first place, the internalization of the sources of financial news, and secondly, some analysis has been made with a natural language processing software, that usually works better with english idioma, instead of italian. In particular, coreferences resolution, sentiment analysis and open information extraction have been applied. \\