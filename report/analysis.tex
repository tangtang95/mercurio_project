\section{Analysis}
As said before, after the data collection process, the information has been analyzed and manipulated in different ways that are going to be explored in the following pages. In order to accomplish this task,
Stanford CoreNLP \cite{stanfordcorenlp} has been used. In particular, a python package that uses Stanford CoreNLP server, has been adopted \cite{corenlpwrapper}. \\
This natural language software provides a set of human language technology tools that makes very easy to apply linguistic analysis to a piece of text (in this case the financial article). \\
The python wrapper presents a simple API that allows the programmer to call the Stanford CoreNLP annotators \cite{annotators}, which basically process a sentence in a certain language returning a response in a specified output format, e.g.: JSON, XML or text. The following example, performs the sentiment analysis of a given string and stores it into an XML variable.
\begin{minted}[mathescape,
               linenos,
               numbersep=5pt,
               gobble=0,
               frame=lines,
               framesep=2mm]{python}
text = "This is an example"
props = {'annotators': 'sentiment','pipelineLanguage':'en','outputFormat':'xml'}
xml = nlp.annotate(text, properties=props)
nlp.close()
\end{minted}
Among the available types of output, XML has been chosen: it get parsed with Beautiful Soup 4 \cite{bs4}. \par 
Coreference resolution, lemmatization, sentiment analysis and open information extraction, are the main language tools that have been used. As a result of the fact that dcoref, which is the annotator that solves pronominal and nominal coreferences, and lemmatisation take significant amounts of time to run, their manipulated results are saved in two different fields of a new database table. Doing so allow to change the mechanisms of the scripts regarding sentiment analysis and open information extraction in future, and therefore re-run them without the time waste mentioned above.
\subsection{Coreference resolution}
The coreference occurs when two or more expression inside a text refers to the same object (e.g. "Jessica has sold her phone to Marcus": in this phrase the pronoun "her" refers to "Jessica"). 
This analysis is done to capture every coreferences inside the article scraped from the website, with the help of the external library Stanford CoreNLP, and substitute it with the representation of that expression (e.g. Jessica has sold Jessica's phone to Marcus).
\par 
Basically, the algorithm written for this analysis consists in dividing the text in more phrases splitted by a dot, and for each phrase analyze it with CoreNLP. 
After that it has been used BeautifulSoup to analyze the XML which the external library gives back; to find the representation and the coreferences, there is a tag "coreference" inside the XML. 
Since there is also another tag called "sentences" that contains all the text analyzed, so it has been used for the substitution of the pronoun or other literal expression into the representation text. 
Every time a coreference expression is founded, the algorithm will replace that coreference expression (i.e. pronoun or other literal expression) with the representation of that. 
At the end the entire text is reconstructed from the tag XML "sentences" that contains the string of text replaced.
\par
It was also possible to analyze the entire text with CoreNLP, but after some tests it has been founded a lot of errors about the coreference (e.g. replacement of "he" with "Clinton"). 
So it has been decided to split the entire article into different substrings of 5 phrases and with this technique it's possible to at least found coreference inside two or more expression. 

\subsubsection{Saving Data}
After the analysis of coreference the new data needs to be saved somewhere. In this project it has been stored on a database. The tables created for all the study of article content are:
\begin{itemize}
	\item articles\_en\_analyzed: this contains the same field of the old table "articles\_en\_full" but with 5 more fields:
	\begin{enumerate}
		\item coref\_content: a field that contains the new article content with the coreference replaced with its representative
		\item lemma\_content: this contains the text article with every word lemmatized (made by using the lemma annotators)
		\item sentiment: a field that can contains "very negative", "negative", "neutral", "positive" or "very positive". This is useful for the sentiment analysis
		\item sentimentNoNeutral: it's the same of the above one, but in this case the "neutral" is ignored (i.e. the "neutral" sentiment returned by a phrase from CoreNLP is ignored)
		\item sentimentSummarized: it's the same of the first sentiment, but with this, it analyzes a summarized text instead of the entire article
	\end{enumerate}
	\item openie\_reports: this table is linked with the article id of the table above, and it contains 5 fields:
	\begin{enumerate}
		\item reportId: the number identification of the openie\_reports (primary key)
		\item articleId: the article id of the table above (it's a foreign key)
		\item subject: it represents the subject inside a literal expression
		\item verb: it represents the action of the subject
		\item object: it represents the object on which the action is taken
	\end{enumerate}
\end{itemize}
In this part of the project, all the news articles are read from the table "articles\_en\_full", and then after the analysis, the news is saved on the new table "articles\_en\_analyzed" with two new field: coref\_content and lemma\_content.


\subsection{Sentiment analysis}
The purpose of this analysis is to give each article a sentiment, in order to allow pattern mining to forecast financial catastrophes. This approach has been implemented in the english version of the project because of the well-trained nlp libraries available and due to the fact the international press was thought to be more vigorous and less apathetic compared to the italian one. \\
The sentiment tool provided by Stanford CoreNLP parses a sentence and returns "very negative", "negative", "neutral", "positive" or "very positive", depending on the deep learning model \cite{sentimentdeep} used in its implementation. So it's been necessary to find a way of weighting the results associated with each sentence presents in the article. \\
The problem has been approached in three different ways that will be explained in the following pages.
However, every implemented solution is based on the fact that clearly not all of the phrases in an article have the same level of importance, thus some periods should weight more than others. The solution adopted to overcome this problem is creating a lemmatized vocabulary composed of financial nouns and keywords. When parsing a sentence, the more it contains these terms, the more it weights, following a linear relation. \\
\begin{math}
weight = 1 + 0.5*n \\
\end{math}
Where n is the number of words present in the dictionary that appear in the sentence too.
The multiplier coefficient (0.5) has been chosen in an empirical way and can be subject of further research and learning models. \\
The weight is then used to modify a python dictionary which has as keys "very negative", "negative", "neutral", "positive" and "very positive" according to the result given by Stanford CoreNLP sentiment annotator. 

\subsubsection{Sentiment analysis with evaluation on neutral values}
Due to the fact that studies have reported the importance and relevance of neutral value\cite{neutralvalues}, in this first method they are considered not negligible. Furthermore, the whole text of the financial news is considered.\\
So once that the algorithm described above is applied, when it comes to assigning the sentiment of the article, the dictionary comes into play: the aim is to determine a number between 1 and -1 that represents the sentiment of the article (w\textsubscript{a}). The interval [-1,+1] is then divided in five equals parts, each of which is assigned to a sentiment and according to w\textsubscript{a} the article's sentiment is calculated.\\
Let be w\textsubscript{vn}, w\textsubscript{n}, w\textsubscript{ne}, w\textsubscript{p}, w\textsubscript{vp} the weights of "very negative", "negative", "neutral, "positive" and "very positive" for the whole article respectively, and m\textsubscript{vn}, m\textsubscript{n}, m\textsubscript{ne}, m\textsubscript{p} and m\textsubscript{vp} their central position in the [-1,+1] interval. Being i the position of the sentiment mentioned above:\newline \newline
\begin{math}
i \in N, i \in [0,4] \\
\end{math}
\begin{equation}
m_i= -0.8+i*0.4
\end{equation}
\begin{equation}
w_a = \frac{\sum\limits_{i} w_i * m_i}{\sum\limits_{i} w_i}
\end{equation}
Doing so, some important properties are guaranteed: neutral values are relevant and w\textsubscript{a} gets closer to the most valued voice of the dictionary. An important note is that this approach won't permit to compare different level of positiveness and negativeness. \\
Nevertheless, once that the script's been run on the server, the results didn't match the expectations: the majority, or, for better saying, the almost totality, of the articles were valued as "neutral" due to the fact that a huge number of phrases contained were analyzed "neutral" as well from NLP. This is the main reason that lead the the develop of the second and the third method.

\subsubsection{Sentiment analysis neglecting neutral values}

The second method that's been studied neglects the presence of neutral values. The only thing that is going to change w.r.t. the precedent solution, is 
\begin{equation}
w_a = \frac{\sum\limits_{i} w_i * m_i}{\sum\limits_{i} w_i} 
\end{equation}
In this case in the denominator the summation doesn't involve w\textsubscript{ne}. In this way, the huge amount of weight of neutral sentences is nullified, thus making the algorithm much more susceptible to the presence of positive or neutral phrases.\\

\subsubsection{Sentiment analysis of summarized text}
The last solution makes use of a summarization script, which was exploited in the italian version of Mercurio and that has been adjusted for the english version just modifying a list of stop words. \\
Due to the fact that a summary contains the most important sentences of an article, analyzing the abstract of a financial news by removing the most of the content could benefit the result because, as already said, not all of the phrases present in it weights the same. \\
The summarization has been combined with the strategy described in the first method. 

\subsubsection{Considerations on sentiment analysis}
Querying the database it is possible to infer some conclusions. The efforts for making the algorithm more susceptible to positive and negative phrases were totally useless: all the row of the three columns saved in the database reports only "neutral" result. This is due to the fact that the initial hypothesis that american press isn't apathetic is basically false. In fact, the script has been launched printing the sentiment of each analyzed sentence, and they are all reported as "neutral". This is a clearly problem for the entire process. \\
Furthermore, independently to the considered method, there is a problem with the whole algorithm that regards linguistic itself. Let's consider the phrase "Facebook is near to declare bankrupt due to the recent scandal. Its competitors can take a big advantage from this.". The first period is valued "negative", while the latter "positive", so a further implementation has to consider this fact: the sentiment depends even on the subject of the period: some fact could benefit a company at the expanse of its competitor, for instance. An open road could be developing the sentiment analysis for only a small group of companies, and implementing a subject recognizer for the article and the single sentences, in order to calculate in a more precise way w\textsubscript{a}. The focus on a small group of companies allows, furthermore, to associate each firm with its CEO, in a deterministic way.  
\subsection{Open information extraction}
\par 
The Open Information Extraction (OpenIE) annotator extracts open-domain relation triples, representing a subject, a relation, and the object of the relation. This allows to summarize the articles in a series of triples and to evaluate the actions performed by the subjects. 
For example, from the sentence\\
Dante wrote the Divine Comedy.\\
the following triples might be extracted:
\begin{itemize}
\item
(Dante) (wrote) (Comedy)
\end{itemize}
The extraction is said to be a textual representation of a potential fact because its elements are not linked to a knowledge base. Furthermore, the factual nature of the proposition has not yet been established. In the above example, transforming the extraction into a full fledged fact would first require linking, if possible, the relation and the arguments to a knowledge base. Second, the truth of the extraction would need to be determined. In computer science transforming OIE extractions into ontological facts is known as relation extraction.
\par 
The algorithm, in order to work, requires that the articles are first pre-processed, by resolving the co-references so as to extract the triples with the correct subjects. \\
The process begins by asking in input the reference journal from which to retrieve the financial items to be examined, then subdividing the individual articles into short periods that are easier to analyze from the Stanford coreNLP library. The library for each period returns a possible triplet that summarizes the described action, and assigns to the analysis a degree of confidence that goes from 0 to 1, where 1 is the maximum correctness. This is one of the critical points of the algorithm, since there are several solutions that have a degree of confidence at 1, choosing the most correct is a very complex problem that would require a much more specific analysis, which at the moment is impossible to find in any open source library. For this reason simply returns the first triplet with confidence to 1 that it finds.
\begin{minted}[mathescape,
               linenos,
               numbersep=5pt,
               gobble=0,
               frame=lines,
               framesep=2mm]{python}
               
text = "This is an example"
props = {'annotators': 'sentiment','pipelineLanguage':'en','outputFormat':'xml'}
for triple in list_of_solutions.find_all("triple"):
            if triple.get("confidence") == "1.000":
                for text in triple.find_all("text"):
                    triple_save = triple_save + text.string + " "
                if len(triple_save.split()) == 3:
                    return triple_save
\end{minted}

The idea behind this analysis is to evaluate the temperature of the verbs, that refer to the financial sector such as buy, sell, invest etc... found, that is counting the number of times in which the same actions are presented in the various articles and based on their number, to risk a hypothesis on the progress of the subject who performs those actions. If a company buys a lot in the last period, the company is probably doing well.\\
Although the idea was good, this analysis results to have many different problems:
\begin{itemize}
\item
The co-references very often are not correct and the pronouns are not replaced with the correct subjects.
\item
The openIE analysis of the periods often fails to extract a triplet or the result don't have the confidently to 1.
\item
Many verbs extracted from the triples are not contained in the financial dictionary.
\end{itemize}
For all this the analysis on the whole is not very correct.
The library turns out to be much more effective on simple sentences or on articles related to politics, where it manages to study the structure of the sentence and to return satisfactory results. We hope that in the future will come out more powerful open source tools for this type of analysis on natural language.

