\subsection{Open information extraction}
The Open Information Extraction (OpenIE) annotator extracts open-domain relation triples, representing a subject, a relation, and the object of the relation. This allows to summarize the articles in a series of triples and to evaluate the actions performed by the subjects. 
The algorithm, in order to work, requires that the articles are first pre-processed, by resolving the co-references so as to extract the triples with the correct subjects. 
The process begins by asking in input the reference journal from which to retrieve the financial items to be examined, then subdividing the individual articles into short periods that are easier to analyze from the Stanford coreNLP library. The library for each period returns a possible triplet that summarizes the described action, and assigns to the analysis a degree of confidence that goes from 0 to 1, where 1 is the maximum correctness. This is one of the critical points of the algorithm, since there are several solutions that have a degree of confidence at 1, choosing the most correct is a very complex problem that would require a much more specific analysis, which at the moment is impossible to find in any open source library. For this reason simply returns the first triplet with confidence to 1 that it finds.
\begin{minted}[mathescape,
               linenos,
               numbersep=5pt,
               gobble=0,
               frame=lines,
               framesep=2mm]{python}
               
text = "This is an example"
props = {'annotators': 'sentiment','pipelineLanguage':'en','outputFormat':'xml'}
for triple in list_of_solutions.find_all("triple"):
            if triple.get("confidence") == "1.000":
                for text in triple.find_all("text"):
                    triple_save = triple_save + text.string + " "
                if len(triple_save.split()) == 3:
                    return triple_save
\end{minted}

The idea behind this analysis is to evaluate the temperature of the verbs, that refer to the financial sector such as buy, sell, invest etc... found, that is counting the number of times in which the same actions are presented in the various articles and based on their number, to risk a hypothesis on the progress of the subject who performs those actions. If a company buys a lot in the last period, the company is probably doing well.
Although the idea was good, this analysis results to have many different problems:
\begin{itemize}
\item
The co-references very often are not correct and the pronouns are not replaced with the correct subjects.
\item
The openIE analysis of the periods often fails to extract a triplet or the result don't have the confidently to 1.
\item
Many verbs extracted from the triples are not contained in the financial dictionary.
\end{itemize}
For all this the analysis on the whole is not very correct.
The library turns out to be much more effective on simple sentences or on articles related to politics, where it manages to study the structure of the sentence and to return satisfactory results. We hope that in the future will come out more powerful open source tools for this type of analysis on natural language.
